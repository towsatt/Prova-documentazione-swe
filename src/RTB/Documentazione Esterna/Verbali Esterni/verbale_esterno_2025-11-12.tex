
% -------------------------------------------------------------------
%  SEZIONE: configurazione.tex
% -------------------------------------------------------------------
\documentclass[a4paper, 11pt, oneside]{scrartcl} % Classe KOMA-Script

% --- Pacchetti Fondamentali ---
\usepackage[utf8]{inputenc}     % Codifica UTF-8
\usepackage[T1]{fontenc}        % Font encoding moderno
\usepackage[italian]{babel}     % Lingua italiana
\usepackage{lmodern}            % Font "Latin Modern"

% --- Grafica e Layout ---
\usepackage{graphicx}           % Per le immagini
\graphicspath{{../../../assets/}} 
\usepackage[a4paper, top=2.5cm, bottom=3cm, left=2.5cm, right=2.5cm]{geometry} % Margini
\usepackage{fancyhdr}           % Per header e footer personalizzati
\usepackage{microtype}          % Migliora la tipografia
\usepackage[svgnames]{xcolor}   % Colori

% --- Utility ---
\usepackage{booktabs}           % Tabelle più professionali
\usepackage{enumitem}           % Per personalizzare liste
\usepackage{hyperref}           % Rende i link cliccabili
\hypersetup{
    colorlinks=true,
    linkcolor=DarkBlue,
    filecolor=DarkBlue,      
    urlcolor=DarkBlue,
    citecolor=DarkBlue,
    pdftitle={Verbale Esterno - NightPRO},
    pdfauthor={Gruppo NightPRO},
}


% ===================================================================
%  IMPOSTAZIONE HEADER E FOOTER
% ===================================================================
\pagestyle{fancy}
\fancyhf{} % Pulisce tutti i campi
\fancyhead[L]{NightPRO - Progetto Ingegneria del Software}
\fancyhead[R]{Anno Accademico 2025/2026}
\fancyfoot[C]{\thepage} % Numero di pagina al centro in basso
\renewcommand{\headrulewidth}{0.4pt} % Linea sottile sotto l'header
\renewcommand{\footrulewidth}{0pt}

% ===================================================================
%  INIZIO DEL DOCUMENTO
% ===================================================================
\begin{document}

% -------------------------------------------------------------------
%  SEZIONE: intestazione_titolo.tex
% -------------------------------------------------------------------
\thispagestyle{empty}
\begin{titlepage}
    \centering
    
\begin{figure}
    \centering
    \includegraphics[width=0.4\textwidth]{logo.png} 
\end{figure}

    \vfill
    
    {\small UNIVERSITÀ DEGLI STUDI DI PADOVA \par}
    {\small CORSO DI LAUREA IN INFORMATICA (L-31) \par}
    \vspace{0.5cm}
    {\large Corso di Ingegneria del Software \par}
    {\small Anno Accademico 2025/2026 \par}
    
    \vfill
    
    {\Huge \bfseries Verbale di Riunione \par}
    
    \vspace{1cm}
    
    % Titolo specifico del documento
    {\Large \itshape Verbale Esterno del 12 Novembre 2025 \par} 
    
    \vfill
    
    {\Large \bfseries Gruppo: NightPRO \par}
    \vspace{0.5cm}
    {\large \href{mailto:swe.nightpro@gmail.com}{swe.nightpro@gmail.com} \par}
    
    \vfill
 
    % Data di redazione
    {\large Data: 2025-11-12 \par}

\end{titlepage}

% -------------------------------------------------------------------
%  SEZIONE: indice.tex
% -------------------------------------------------------------------
\newpage
\tableofcontents % Genera l'indice
\pagestyle{fancy} % Riattiva lo stile di pagina da qui in poi

% -------------------------------------------------------------------
%  SEZIONE: informazioni.tex
% -------------------------------------------------------------------
\newpage
\section{Informazioni Generali}

\subsection{Componenti del Gruppo}
Elenco dei membri del gruppo di lavoro NightPRO.
\begin{table}[h!]
\centering
\begin{tabular}{@{}llc@{}}
\toprule
\textbf{Cognome} & \textbf{Nome} & \textbf{Matricola} \\
\midrule
Biasuzzi & Davide & 2111000 \\
Bilato & Leonardo & 2071084 \\
Zanella & Francesco & 2116442 \\
Romascu & Mihaela-Mariana & 2079726 \\
Ogniben & Michele & 2042325 \\
Perozzo & Samuele & 2110989 \\
Ponso & Giovanni & 2000558 \\
\bottomrule
\end{tabular}
\caption{Componenti del Gruppo NightPRO.}
\end{table}

\subsection{Dettagli Riunione}
\begin{itemize}
    \item \textbf{Data:} 2025-11-12
    \item \textbf{Ora:} 16:30 - 17:00
    \item \textbf{Luogo:} Google Meet
    \item \textbf{Partecipanti (NightPRO):} Biasuzzi Davide, Bilato Leonardo, Zanella Francesco, Romascu Mihaela-Mariana, Perozzo Samuele, Ponso Giovanni
    \item \textbf{Partecipanti (Esterni):} Gianluca Carlesso (Ergon Informatica)
    \item \textbf{Redatto da: } Davide Biasuzzi
    \item \textbf{Verificato da:} Giovanni Ponso
    \item \textbf{Versione: } 1.0
\end{itemize}


% -------------------------------------------------------------------
%  SEZIONE: odg.tex (Ordine del Giorno)
% -------------------------------------------------------------------
\newpage
\section{Ordine del Giorno (Agenda)}
\begin{itemize}
    \item[1.] Definizione delle modalità di comunicazione e collaborazione.
    \item[2.] Chiarimenti sui requisiti tecnici e funzionali del Proof of Concept (PoC).
    \item[3.] Discussione sullo stack tecnologico (LLM, Webapp, DB).
    \item[4.] Approfondimenti su input (audio, WhatsApp) e output (JSON).
    \item[5.] Gestione della sicurezza e della privacy dei dati.
\end{itemize}

% -------------------------------------------------------------------
%  SEZIONE: diario.tex (Diario della riunione)
% -------------------------------------------------------------------
\newpage
\section{Diario della Riunione}
Discussione con il referente di Ergon Informatica, Gianluca Carlesso, per definire i requisiti e le modalità di lavoro del progetto. La discussione è stata strutturata come segue:

\footnotesize  % Font più piccolo
\setlength{\tabcolsep}{8pt}  % Spazio tra le colonne
\begin{tabular}{p{0.45\textwidth}p{0.45\textwidth}}
\toprule
\textbf{Argomento/Domanda} & \textbf{Risposta/Discussione} \\
\midrule
\textbf{Modalità di Comunicazione} \par
Come gestiamo le comunicazioni sincrone e asincrone? Ogni quanto possiamo fissare riunioni?
&
Per le comunicazioni asincrone si possono usare la mail o Telegram di Gianluca Carlesso. Per le riunioni sincrone si usa Google Meet. È possibile fissare riunioni in base alle esigenze del gruppo, senza una cadenza fissa.
\\ \addlinespace
\textbf{Notifica Operatore} \par
Come è possibile avvisare un operatore umano quando il sistema non riesce a processare un ordine?
&
È possibile inviare una mail (configurabile nelle impostazioni) che segnali l'errore. Sarebbe utile allegare alla mail il messaggio originale (testo, audio o immagine) e identificare il cliente.
\\ \addlinespace
\textbf{Gestione Utenti e Login} \par
L'identificativo utente è gestito da noi? L'utente avrà un account? Serve una registrazione?
&
Ci si può appoggiare a un sistema di autenticazione già esistente, ma il login va gestito. L'azienda può includere degli utenti di esempio nel dataset. La registrazione nell'app è un requisito opzionale, da integrare in futuro su un sistema che ha già la registrazione.
\\ \addlinespace
\textbf{Dati e Caso Studio} \par
È possibile ricevere dati di un caso studio per farci un'idea?
&
Sì, l'azienda (G. Carlesso) può fornire un caso studio nei primi giorni della prossima settimana.
\\ \addlinespace
\textbf{Struttura Output (JSON)} \par
Qual è la struttura esatta (schema) dell'ordine che il PoC dovrà generare?
&
Il sistema deve creare un file JSON. L'azienda fornirà lo schema esatto da utilizzare per la strutturazione del JSON.
\\ \addlinespace
\textbf{Database Interno e Logging} \par
Possiamo avere un nostro DB interno?
&
Sì. Si può usare per il logging: tenere traccia dell'input dell'utente e dell'output generato (per debug e tracciabilità). L'ordine finale creato va depositato in un path esterno condiviso.
\\ \addlinespace
\textbf{Priorità Input (PoC)} \par
Le modalità prioritarie per il PoC sono audio e testo?
&
Sì, il testo è la base. La scelta migliore è sviluppare la funzionalità di estrazione dati dal testo e poi implementare la conversione Audio-Testo, per riutilizzare la stessa logica.
\\ \addlinespace
\textbf{Metriche di Successo (PoC)} \par
Quali metriche ci consiglia di usare per dimostrare il successo del PoC?
&
Si può partire da ordini reali, estrarre il testo originale e misurare quanti articoli (identificati da codice articolo e quantità) il sistema riesce a individuare correttamente.
\\ \addlinespace
\textbf{Consigli Generali} \par
Errori comuni? Approccio?
&
Gianluca consiglia di non strafare e di procedere a step, perché il tempo è poco. Partire con i requisiti obbligatori e integrare quelli desiderabili e opzionali solo quando il sistema è stabile.
\\ \addlinespace
\textbf{Tecnologie: WebApp} \par
Framework per la webapp?
&
Sì, usare React. È un framework che piace all'azienda.
\\ \addlinespace
\textbf{Tecnologie: LLM} \par
GPT o OLLAMA?
&
GPT ha una maggiore facilità di integrazione. L'azienda fornirà una chiave di licenza. OLLAMA è un'alternativa, ma richiederebbe di farlo girare in locale.
\\ \addlinespace
\textbf{Input Audio: Limiti} \par
Ci sono limiti o formati richiesti per i messaggi vocali?
&
Le librerie (es. Google) accettano quasi ogni formato. L'unica difficoltà reale è data da audio con molto rumore di fondo, anche se le librerie moderne effettuano già una buona pulizia.
\\ \addlinespace
\textbf{Input: Integrazione WhatsApp} \par
Come possiamo interfacciarci con WhatsApp?
&
Se il gruppo vuole implementare l'integrazione WhatsApp, l'azienda può fornire aiuto. Non è difficile, ma richiede "un giro un po' complesso".
\\ \addlinespace
\textbf{Sicurezza e Privacy} \par
Ci sono norme da seguire per i dati? Serve una privacy policy? I dati vanno cancellati?
&
Gli utenti devono autenticarsi. La comunicazione web deve avvenire tramite HTTPS.
Non serve implementare una privacy policy (l'azienda gestisce già la cybersecurity e i dati).
I dati NON devono essere cancellati dopo un "tot" tempo, ma devono essere memorizzati in archiviazione sostitutiva (se ne occupa l'azienda).
\\
\bottomrule
\end{tabular}
\normalsize

% -------------------------------------------------------------------
%  SEZIONE: decisioni.tex (Decisioni prese)
% -------------------------------------------------------------------
\newpage
\section{Decisioni Prese}

\begin{enumerate}
    \item \textbf{Comunicazioni:} Asincrone via Telegram/Email; Sincrone via Google Meet su richiesta.
    \item \textbf{Autenticazione:} Si gestirà il login appoggiandosi a sistemi esistenti. La registrazione è requisito opzionale.
    \item \textbf{Dati:} L'azienda fornirà un caso studio d'esempio e lo schema JSON per l'output.
    \item \textbf{Architettura PoC:} Si utilizzerà un DB interno per il logging. L'output (ordine JSON) sarà salvato su un path condiviso.
    \item \textbf{Priorità Input PoC:} Si implementeranno Testo e Audio (convertito in testo).
    \item \textbf{Privacy:} Non è richiesta la gestione della privacy policy o dell'archiviazione sostitutiva (gestite da Ergon). È obbligatorio l'uso di HTTPS.
\end{enumerate}

% -------------------------------------------------------------------
%  SEZIONE: todo.tex (Attività da svolgere)
% -------------------------------------------------------------------
\newpage
\section{Attività da Svolgere (To-Do)}

\begin{table}[h!]
\centering
\begin{tabular}{@{}lll@{}}
\toprule
\textbf{Attività} & \textbf{Assegnatario/i} & \textbf{Scadenza} \\
\midrule
Stesura verbale riunione 12/11 & Davide Biasuzzi & 2025-11-12 \\
Attendere schema JSON e caso studio & \textit{Ergon Informatica} & N/A \\
Valutare quale LLM usare & Gruppo NightPRO & N/A \\
\bottomrule
\end{tabular}
\caption{Riepilogo task assegnati.}
\end{table}
\end{document}