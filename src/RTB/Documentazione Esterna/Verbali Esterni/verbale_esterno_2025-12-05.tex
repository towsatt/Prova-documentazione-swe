\documentclass[a4paper, 11pt, oneside]{scrartcl} % Classe KOMA-Script

% --- Pacchetti Fondamentali ---
\usepackage[utf8]{inputenc}     % Codifica UTF-8
\usepackage[T1]{fontenc}        % Font encoding moderno
\usepackage[italian]{babel}     % Lingua italiana
\usepackage{lmodern}            % Font "Latin Modern"

% --- Grafica e Layout ---
\usepackage{graphicx}           % Per le immagini
\graphicspath{{assets/}{../assets/}{../../assets/}{../../../assets}}
\usepackage[a4paper, top=2.5cm, bottom=3cm, left=2.5cm, right=2.5cm]{geometry} % Margini
\usepackage{fancyhdr}           % Per header e footer personalizzati
\usepackage{microtype}          % Migliora la tipografia
\usepackage[svgnames]{xcolor}   % Colori

% --- Utility ---
\usepackage{booktabs}           % Tabelle più professionali
\usepackage{enumitem}           % Per personalizzare liste
\usepackage{hyperref}           % Rende i link cliccabili
\hypersetup{
    colorlinks=true,
    linkcolor=DarkBlue,
    filecolor=DarkBlue,      
    urlcolor=DarkBlue,
    citecolor=DarkBlue,
    pdftitle={Verbale Esterno - NightPRO},
    pdfauthor={Gruppo NightPRO},
}


% ===================================================================
%  IMPOSTAZIONE HEADER E FOOTER
% ===================================================================
\pagestyle{fancy}
\fancyhf{} % Pulisce tutti i campi
\fancyhead[L]{\textbf{NightPRO – Progetto Ingegneria del Software}}
\fancyhead[R]{Anno Accademico 2025/2026}
\fancyfoot[C]{\thepage} % Numero di pagina al centro in basso
\renewcommand{\headrulewidth}{0.4pt} % Linea sottile sotto l'header
\renewcommand{\footrulewidth}{0pt}

% ===================================================================
%  INIZIO DEL DOCUMENTO
% ===================================================================
\begin{document}

% -------------------------------------------------------------------
%  SEZIONE: intestazione_titolo.tex
% -------------------------------------------------------------------
\thispagestyle{empty}
\begin{titlepage}
    \centering
    
\begin{figure}
    \centering
    \includegraphics[width=0.4\textwidth]{logo.png} 
\end{figure}

    \vfill
    
    {\small UNIVERSITÀ DEGLI STUDI DI PADOVA \par}
    {\small CORSO DI LAUREA IN INFORMATICA (L-31) \par}
    \vspace{0.5cm}
    {\large Corso di Ingegneria del Software \par}
    {\small Anno Accademico 2025/2026 \par}
    
    \vfill
    
    {\Huge \bfseries Verbale di Riunione \par}
    
    \vspace{1cm}
    
    % Titolo specifico del documento
    {\Large \itshape Verbale Esterno del 5 Dicembre 2025 \par} 
    
    \vfill
    
    {\Large \bfseries Gruppo: NightPRO \par}
    \vspace{0.5cm}
    {\large \href{mailto:swe.nightpro@gmail.com}{swe.nightpro@gmail.com} \par}
    
    \vfill
 
    % Data di redazione
    {\large Data: 2025-12-05 \par}

\end{titlepage}

% -------------------------------------------------------------------
%  SEZIONE: indice.tex
% -------------------------------------------------------------------
\newpage
\tableofcontents % Genera l'indice
\pagestyle{fancy} % Riattiva lo stile di pagina da qui in poi

% -------------------------------------------------------------------
%  SEZIONE: informazioni.tex
% -------------------------------------------------------------------
\newpage
\section{Informazioni Generali}

\subsection{Componenti del Gruppo}
Elenco dei membri del gruppo di lavoro NightPRO.
\begin{table}[h!]
\centering
\begin{tabular}{@{}llc@{}}
\toprule
\textbf{Cognome} & \textbf{Nome} & \textbf{Matricola} \\
\midrule
Biasuzzi & Davide & 2111000 \\
Bilato & Leonardo & 2071084 \\
Zanella & Francesco & 2116442 \\
Romascu & Mihaela-Mariana & 2079726 \\
Ogniben & Michele & 2042325 \\
Perozzo & Samuele & 2110989 \\
Ponso & Giovanni & 2000558 \\
\bottomrule
\end{tabular}
\caption{Componenti del Gruppo NightPRO.}
\end{table}

\subsection{Dettagli Riunione}
\begin{itemize}
    \item \textbf{Data:} 2025-12-05
    \item \textbf{Ora:} 09:00 - 09:30
    \item \textbf{Luogo:} Google Meet
    \item \textbf{Partecipanti (NightPRO):} Biasuzzi Davide, Bilato Leonardo, Ogniben Michele, Perozzo Samuele, Ponso Giovanni
    \item \textbf{Partecipanti (Esterni):} Gianluca Carlesso (Ergon Informatica)
    \item \textbf{Redatto da: } Bilato Leonardo
    \item \textbf{Verificato da:} Giovanni Ponso
    \item \textbf{Versione: } 1.0
\end{itemize}


% -------------------------------------------------------------------
%  SEZIONE: odg.tex (Ordine del Giorno)
% -------------------------------------------------------------------
\newpage
\section{Ordine del Giorno (Agenda)}
\begin{itemize}
    \item[1.] Chiarimenti sulla necessità della parte client nella WebApp.
    \item[2.] Chiarimenti sui canali usati dagli utenti per l'invio degli ordini.
    \item[3.] Approfondimento sulle regole aziendali mezionate nel capitolato.
    \item[4.] Chiarimenti su chi sarà il cliente finale del software.
    \item[5.] Chiarimenti su funzionalità necessarie nel PoC
    \item[6.] Richiesta una riunione di design thinking.
    \item[7.] Discussione su quale llm utilizzare.
    \item[8.] Richiesta ad Ergon di esempi d'ordine.
\end{itemize}

% -------------------------------------------------------------------
% SEZIONE: diario.tex (Diario della riunione)
% -------------------------------------------------------------------
\newpage
\section{Diario della Riunione}
Discussione con il referente di Ergon Informatica, Gianluca Carlesso, per chiarire dubbi sorti durante l'analisi dei requisiti. La discussione è stata strutturata come segue:

\vspace{1cm}
\footnotesize
\setlength{\tabcolsep}{8pt}
\begin{tabular}{p{0.45\textwidth}p{0.45\textwidth}}
\toprule
\textbf{Argomento/Domanda} & \textbf{Risposta/Discussione} \\
\midrule

\textbf{1) Interfaccia Utente e Operatore} \par
È necessaria una parte client dedicata? Qual è il flusso previsto per utente e operatore?
&
L’utente finale interagisce tramite una chat con LLM (testo, audio, immagini facoltative).
L’ordine generato viene mostrato all’utente, che può confermare, risolvere ambiguità o chiedere assistenza.
L’operatore ha un'interfaccia separata per la gestione e correzione degli ordini, e interviene in caso di errori o incompletezze.
\\ \addlinespace

\textbf{2) Canali di comunicazione} \par
Quali canali sono previsti per la comunicazione con l'utente?
&
È previsto l'uso di un'interfaccia lato client in stile chatbot.
Canali alternativi (Telegram bot, email, ecc.) sono facoltativi.
\\ \addlinespace

\textbf{3) Regole Aziendali} \par
Quali vincoli aziendali deve rispettare il sistema di generazione ordini?
&
Basarsi sull’anagrafica articoli (codice, descrizione, categorie, U.M.).
L’unità di misura va convertita in colli.
Non tutti i clienti possono acquistare l’intero catalogo: alcuni articoli devono essere filtrati in base al cliente.
\\ \addlinespace

\textbf{4) Cliente Finale} \par
Chi utilizzerà il sistema?
&
Locali come bar, ristoranti e chioschi.
\\ \addlinespace

\textbf{5) PoC: Funzionalità Richieste} \par
Cosa deve dimostrare il Proof of Concept?
&
Dimostrare che da testo (e opzionalmente audio) è possibile generare correttamente un ordine strutturato.
\\ \addlinespace

\textbf{6) Richiesta di riunione di Design Thinking} \par
Richiesta la possibilità di organizzare una sessione di Design Thinking per definire meglio i requisiti utente.
&
L'azienda accetta. Si organizzerà un incontro la settimana successiva per la sessione di Design Thinking.
\\ \addlinespace

\textbf{7) Tecnologie LLM} \par
Quali modelli conviene usare? Ci potete fornire una chiave?
&
Consigliato usare direttamente le API OpenAI. Ergon fornirà le chiavi per accedere al servizio.
\\ \addlinespace

\textbf{8) Template Struttura Ordini} \par
È possibile avere un template JSON della struttura degli ordini?
&
Ergon fornirà esempi di file/strutture per definire il formato dell’ordine.
\\ \addlinespace

\bottomrule
\end{tabular}
\normalsize

% -------------------------------------------------------------------
% SEZIONE: decisioni.tex (Decisioni prese)
% -------------------------------------------------------------------
\newpage
\section{Decisioni Prese}

\begin{enumerate}
\item \textbf{Interfaccia e Flusso Utente:}
\begin{itemize}
\item L’utente finale interagisce con la WebApp tramite un'interfaccia chatbot (testo, audio, immagini facoltative) per la creazione, conferma o risoluzione di ambiguità degli ordini.

\item L’operatore utilizzerà un'interfaccia separata per la gestione e correzione degli ordini.

\item L'implementazione di canali alternativi (es. Telegram, email) è facoltativa.

\end{itemize}




\item \textbf{Validazione e Regole Aziendali (Categoria: Requisiti Funzionali):}
\begin{itemize}
\item Il sistema deve garantire che l'ordine generato sia coerente e valido, basandosi sui dati aziendali (anagrafica, U.M., ecc.).
\item Deve essere gestita la disponibilità del catalogo prodotti, filtrando gli articoli anche in base alle restrizioni del singolo cliente.
\item Devono essere rispettate le unità di misura previste per l'articolo specifico.
\end{itemize}







\item \textbf{PoC (Proof of Concept)} si concentrerà sulla dimostrazione della fattibilità del progetto:

\begin{itemize}
\item Generazione di un ordine strutturato partendo da input testuale.

\item Grezza implementazione dell'interfaccia utente.

\item L'interfaccia operatore deve permettere la gestione degli ordini (conferma e modifica).
\end{itemize}

\item \textbf{Forniture da Ergon:}

\begin{itemize}
\item Ergon fornirà le chiavi API per l'utilizzo del modello OpenAI LLM.

\item Ergon fornirà esempi di input e il template/struttura (es. JSON) da utilizzare per l'ordine generato.

\end{itemize}

\item \textbf{Prossimi Passi:}
\begin{itemize}
\item Organizzazione di una sessione di Design Thinking (Metodologia) per la corretta strutturazione della raccolta requisiti.

\end{itemize}
\end{enumerate}

% -------------------------------------------------------------------
%  SEZIONE: todo.tex (Attività da svolgere)
% -------------------------------------------------------------------
\newpage
\section{Attività da Svolgere (To-Do)}

\begin{table}[h!]
\centering
\begin{tabular}{@{}lll@{}}
\toprule
\textbf{Attività} & \textbf{Assegnatario/i} & \textbf{Scadenza} \\
\midrule
Stesura verbale riunione 05/12 & Bilato Leonardo & 2025-12-05 \\
Ricezione esempi input e struttura ordine & Ergon Informatica & N/A \\
Ricezione chiave llm OpenAi & Ergon Informatica & N/A \\
Organizzazione sessione di design thinking & Ergon + Gruppo NightPRO & Settimana successiva \\
\bottomrule
\end{tabular}
\caption{Riepilogo task assegnati.}
\end{table}

\end{document}