\documentclass[a4paper, 11pt, oneside]{scrartcl} % Classe KOMA-Script

% --- Pacchetti Fondamentali ---
\usepackage[utf8]{inputenc}     % Codifica UTF-8
\usepackage[T1]{fontenc}        % Font encoding moderno
\usepackage[italian]{babel}     % Lingua italiana
\usepackage{lmodern}            % Font "Latin Modern"

% --- Grafica e Layout ---
\usepackage{graphicx}           % Per le immagini
\graphicspath{{assets/}{../assets/}{../../assets/}{../../../assets}}
\usepackage[a4paper, top=2.5cm, bottom=3cm, left=2.5cm, right=2.5cm]{geometry} % Margini
\usepackage{fancyhdr}           % Per header e footer personalizzati
\usepackage{microtype}          % Migliora la tipografia
\usepackage[svgnames]{xcolor}   % Colori

% --- Utility ---
\usepackage{booktabs}           % Tabelle più professionali
\usepackage{enumitem}           % Per personalizzare liste
\usepackage{hyperref}           % Rende i link cliccabili
\hypersetup{
    colorlinks=true,
    linkcolor=DarkBlue,
    filecolor=DarkBlue,      
    urlcolor=DarkBlue,
    citecolor=DarkBlue,
    pdftitle={Verbale Esterno - NightPRO},
    pdfauthor={Gruppo NightPRO},
}


% ===================================================================
%  IMPOSTAZIONE HEADER E FOOTER
% ===================================================================
\pagestyle{fancy}
\fancyhf{} % Pulisce tutti i campi
\fancyhead[L]{\textbf{NightPRO – Progetto Ingegneria del Software}}
\fancyhead[R]{Anno Accademico 2025/2026}
\fancyfoot[C]{\thepage} % Numero di pagina al centro in basso
\renewcommand{\headrulewidth}{0.4pt} % Linea sottile sotto l'header
\renewcommand{\footrulewidth}{0pt}

% ===================================================================
%  INIZIO DEL DOCUMENTO
% ===================================================================
\begin{document}

% -------------------------------------------------------------------
%  SEZIONE: intestazione_titolo.tex
% -------------------------------------------------------------------
\thispagestyle{empty}
\begin{titlepage}
    \centering
    
\begin{figure}
    \centering
    \includegraphics[width=0.4\textwidth]{logo.png} 
\end{figure}

    \vfill
    
    {\small UNIVERSITÀ DEGLI STUDI DI PADOVA \par}
    {\small CORSO DI LAUREA IN INFORMATICA (L-31) \par}
    \vspace{0.5cm}
    {\large Corso di Ingegneria del Software \par}
    {\small Anno Accademico 2025/2026 \par}
    
    \vfill
    
    {\Huge \bfseries Verbale di Riunione \par}
    
    \vspace{1cm}
    
    {\Large \itshape Verbale Esterno dell'11 Dicembre 2025 \par} 
    
    \vfill
    
    {\Large \bfseries Gruppo: NightPRO \par}
    \vspace{0.5cm}
    {\large \href{mailto:swe.nightpro@gmail.com}{swe.nightpro@gmail.com} \par}
    
    \vfill
 
    {\large Data: 2025-12-11 \par}

\end{titlepage}

% -------------------------------------------------------------------
%  SEZIONE: indice.tex
% -------------------------------------------------------------------
\newpage
\tableofcontents % Genera l'indice
\pagestyle{fancy} % Riattiva lo stile di pagina da qui in poi

% -------------------------------------------------------------------
%  SEZIONE: informazioni.tex
% -------------------------------------------------------------------
\newpage
\section{Informazioni Generali}

\subsection{Componenti del Gruppo}
Elenco dei membri del gruppo di lavoro NightPRO.
\begin{table}[h!]
\centering
\begin{tabular}{@{}llc@{}}
\toprule
\textbf{Cognome} & \textbf{Nome} & \textbf{Matricola} \\
\midrule
Biasuzzi & Davide & 2111000 \\
Bilato & Leonardo & 2071084 \\
Zanella & Francesco & 2116442 \\
Romascu & Mihaela-Mariana & 2079726 \\
Ogniben & Michele & 2042325 \\
Perozzo & Samuele & 2110989 \\
Ponso & Giovanni & 2000558 \\
\bottomrule
\end{tabular}
\caption{Componenti del Gruppo NightPRO.}
\end{table}

\subsection{Dettagli Riunione}
\begin{itemize}
    \item \textbf{Data:} 2025-12-11
    \item \textbf{Ora:} 15:15 - 16:00
    \item \textbf{Luogo:} Google Meet
    \item \textbf{Partecipanti (NightPRO):} Bilato Leonardo, Perozzo Samuele, Romascu Mihaela-Mariana, Zanella Francesco
    \item \textbf{Partecipanti (Esterni):} Gianluca Carlesso (Ergon Informatica)
    \item \textbf{Redatto da: } Perozzo Samuele
    \item \textbf{Verificato da:} Romascu Mihaela-Mariana
    \item \textbf{Versione: } 1.0
\end{itemize}


% -------------------------------------------------------------------
%  SEZIONE: odg.tex (Ordine del Giorno)
% -------------------------------------------------------------------
\newpage
\section{Ordine del Giorno (Agenda)}
\begin{itemize}
    \item[1.] Revisione dell'Analisi dei Requisiti (Attori e Use Cases).
    \item[2.] Definizione delle metriche di affidabilità per PoC e Prodotto Finale.
    \item[3.] Gestione delle ambiguità e intervento dell'operatore.
    \item[4.] Definizione degli attributi obbligatori per l'ordine.
    \item[5.] Discussione sullo stack tecnologico e coordinamento con altri team.
\end{itemize}

% -------------------------------------------------------------------
% SEZIONE: diario.tex (Diario della riunione)
% -------------------------------------------------------------------
\newpage
\section{Diario della Riunione}
Durante l'incontro con il referente di Ergon Informatica, Gianluca Carlesso, sono stati discussi i dettagli relativi all'Analisi dei Requisiti e alla strutturazione tecnica del progetto.

\vspace{1cm}
\footnotesize
\setlength{\tabcolsep}{8pt}
\begin{tabular}{p{0.45\textwidth}p{0.45\textwidth}}
\toprule
\textbf{Argomento/Domanda} & \textbf{Risposta/Discussione} \\
\midrule

\textbf{1) Correttezza Attori e Use Case} \par
Abbiamo presentato la nostra struttura attuale degli attori e dei casi d'uso chiedendo conferma sulla loro validità.
&
L'azienda consiglia di \textbf{suddividere ulteriormente gli Use Case}. Attualmente risultano troppo onnicomprensivi e racchiudono troppe funzionalità in un singolo caso. Una maggiore granularità favorirà chiarezza e manutenzione.
\\ \addlinespace

\textbf{2) Affidabilità del Sistema} \par
Quali sono le metriche di affidabilità attese per la generazione degli ordini?
&
Il prodotto finale dovrà garantire un'affidabilità (correttezza dell'ordine generato rispetto all'input) pari al \textbf{70\%}.
Per il Proof of Concept (PoC) è accettabile una soglia indicativa del \textbf{50\%}.
\\ \addlinespace

\textbf{3) Gestione delle Ambiguità} \par
Quali sono i casi comuni di ambiguità e come dobbiamo gestirli se l'IA non riesce a risolvere il problema con l'utente?
&
Nel caso in cui si verifichi uno stallo o un'ambiguità non risolvibile tra IA e Utente, è consigliato implementare nella WebApp un \textbf{bottone "Chiama Operatore"} (o simile) per richiedere l'intervento umano diretto.
\\ \addlinespace

\textbf{4) Attributi dell'Ordine} \par
Quali sono i dati minimi indispensabili per considerare valido un ordine?
&
Gli attributi obbligatori sono il \textbf{Codice Prodotto} e una \textbf{Breve Descrizione} dell'articolo.
\\ \addlinespace

\textbf{5) Tecnologie e Collaborazione} \par
Discussione sulle difficoltà riscontrate con React e proposta di allineamento tecnologico.
&
Ergon ha proposto un incontro di conoscenza e apprendimento con l'altro team che lavora al medesimo progetto.
L'obiettivo è valutare l'uniformità dello stack tecnologico: \textbf{Docker, FastAPI e React}.
È stato proposto un meeting congiunto per il \textbf{12 gennaio alle ore 14:00}.
\\ \addlinespace

\bottomrule
\end{tabular}
\normalsize

% -------------------------------------------------------------------
% SEZIONE: decisioni.tex (Decisioni prese)
% -------------------------------------------------------------------
\newpage
\section{Decisioni Prese}

\begin{enumerate}
\item \textbf{Rifattorizzazione Use Case:}
\begin{itemize}
\item Si procederà alla decomposizione degli attuali Use Case in unità più piccole e specifiche per migliorare la qualità dell'analisi dei requisiti.
\end{itemize}

\item \textbf{Obiettivi di Qualità (Affidabilità):}
\begin{itemize}
\item Target PoC: $\approx 50\%$ di accuratezza.
\item Target Prodotto Finale: $70\%$ di accuratezza.
\end{itemize}

\item \textbf{Gestione Eccezioni (UX):}
\begin{itemize}
\item Verrà introdotto un meccanismo (es. pulsante dedicato) nell'interfaccia utente per permettere all'utente di segnalare la necessità di intervento da parte dell'operatore umano in caso di loop o incomprensioni con l'IA.
\end{itemize}

\item \textbf{Struttura Dati Ordine:}
\begin{itemize}
\item Il modello dati dell'ordine dovrà prevedere obbligatoriamente i campi \emph{Codice Prodotto} e \emph{Descrizione}.
\end{itemize}

\item \textbf{Stack Tecnologico e Collaborazione:}
\begin{itemize}
\item Il gruppo NightPRO valuterà internamente l'adozione dello stack completo (Docker + FastAPI) per uniformarsi a React (già scelto).
\end{itemize}
\end{enumerate}

% -------------------------------------------------------------------
%  SEZIONE: todo.tex (Attività da svolgere)
% -------------------------------------------------------------------
\newpage
\section{Attività da Svolgere (To-Do)}

\begin{table}[h!]
\centering
\begin{tabular}{@{}lll@{}}
\toprule
\textbf{Attività} & \textbf{Assegnatario/i} & \textbf{Scadenza} \\
\midrule
Suddivisione e refactoring Use Cases & Analisti & Sprint 5 \\
Valutazione interna adozione FastAPI/Docker & Tutto il gruppo & Sprint 6 \\
Conferma presenza riunione 12 Gennaio & Responsabile & Breve termine \\
Implementazione bozza UI con bottone operatore & Sviluppatori & Fase PoC \\
\bottomrule
\end{tabular}
\caption{Riepilogo task assegnati.}
\end{table}

\end{document}