\documentclass[a4paper, 11pt, oneside]{scrartcl} % Classe KOMA-Script

% --- Pacchetti Fondamentali ---
\usepackage[utf8]{inputenc}      % Codifica UTF-8
\usepackage[T1]{fontenc}         % Font encoding moderno
\usepackage[italian]{babel}      % Lingua italiana 
\usepackage{lmodern}             % Font "Latin Modern"

% --- Grafica e Layout ---
\usepackage{graphicx}            % Per includere immagini
\usepackage{currfile}
\graphicspath{{../../../assets/}}

\usepackage[a4paper, top=2.5cm, bottom=3cm, left=2.5cm, right=2.5cm]{geometry} % Margini
\usepackage{fancyhdr}            % Per header e footer personalizzati
\usepackage{microtype}           % Migliora la tipografia
\usepackage[svgnames]{xcolor}    % Colori

% --- Utility ---
\usepackage{booktabs}            % Tabelle più professionali
\usepackage{enumitem}            % Per personalizzare liste
\usepackage{hyperref}            % Rende i link cliccabili
\hypersetup{
    colorlinks=true,
    linkcolor=DarkBlue,
    filecolor=DarkBlue,     
    urlcolor=DarkBlue,
    citecolor=DarkBlue,
    pdftitle={Documento Progetto - NightPRO},
    pdfauthor={Gruppo NightPRO},
}

% ===================================================================
%  IMPOSTAZIONE HEADER E FOOTER
% ===================================================================
\pagestyle{fancy}
\fancyhf{} % Pulisce tutti i campi
\fancyhead[L]{NightPRO - Progetto Ingegneria del Software}
\fancyhead[R]{Anno Accademico 2025/2026}
\fancyfoot[C]{\thepage} % Numero di pagina al centro in basso
\renewcommand{\headrulewidth}{0.4pt} % Linea sottile sotto l'header
\renewcommand{\footrulewidth}{0pt}

% ===================================================================
%  INIZIO DEL DOCUMENTO
% ===================================================================
\begin{document}

% -------------------------------------------------------------------
%  SEZIONE: intestazione_titolo.tex
% -------------------------------------------------------------------
\thispagestyle{empty}
\begin{titlepage}
    \centering
    
\begin{figure}
    \centering
    \includegraphics[width=0.4\textwidth]{logo.png}
\end{figure}

    \vfill
    
    {\small UNIVERSITÀ DEGLI STUDI DI PADOVA \par}
    {\small CORSO DI LAUREA IN INFORMATICA (L-31) \par}
    \vspace{0.5cm}
    {\large Corso di Ingegneria del Software \par}
    {\small Anno Accademico 2025/2026 \par}
    
    \vfill
    
    {\Huge \bfseries Verbale di Riunione \par}
    
    \vspace{1cm}
    
    {\Large \itshape Verbale Interno del 10 Novembre 2025 \par} 
    
    \vfill
    
    {\Large \bfseries Gruppo: NightPRO \par}
    \vspace{0.5cm}
    {\large \href{mailto:swe.nightpro@gmail.com}{swe.nightpro@gmail.com} \par}
    
    \vfill
 
    {\large Data: 2025-11-10 \par}

\end{titlepage}

% -------------------------------------------------------------------
%  SEZIONE: indice.tex
% -------------------------------------------------------------------
\newpage
\tableofcontents % Genera l'indice
\pagestyle{fancy} % Riattiva lo stile di pagina da qui in poi

% -------------------------------------------------------------------
%  SEZIONE: informazioni.tex
% -------------------------------------------------------------------
\newpage
\section{Informazioni Generali}

\subsection{Componenti del Gruppo}
Elenco dei membri del gruppo di lavoro NightPRO.
\begin{table}[h!]
\centering
\begin{tabular}{@{}llc@{}}
\toprule
\textbf{Cognome} & \textbf{Nome} & \textbf{Matricola} \\
\midrule
Biasuzzi & Davide & 2111000 \\
Bilato & Leonardo & 2071084 \\
Zanella & Francesco & 2116442 \\
Romascu & Mihaela-Mariana & 2079726 \\
Ogniben & Michele & 2042325 \\
Perozzo & Samuele & 2110989 \\
Ponso & Giovanni & 2000558 \\
\bottomrule
\end{tabular}
\caption{Componenti del Gruppo NightPRO.}
\end{table}

\subsection{Dettagli Riunione}
\begin{itemize}
    \item \textbf{Data:} 2025-11-10
    \item \textbf{Ora:} 16:30 - 18:15
    \item \textbf{Luogo:} Google Meet
    \item \textbf{Partecipanti:} Tutti i membri del gruppo tranne Michele Ogniben e Samuele Perozzo
    \item \textbf{Redatto da: } Davide Biasuzzi
    \item \textbf{Verificato da:} Francesco Zanella 
    \item \textbf{Versione: } 1.0
\end{itemize}


% -------------------------------------------------------------------
%  SEZIONE: odg.tex (Ordine del Giorno)
% -------------------------------------------------------------------
\newpage
\section{Ordine del Giorno (Agenda)}
\begin{itemize}
    \item[1.] Discussione sul Versionamento
    \item[2.] Definizione prossimi passi verso la RTB
\end{itemize}

% -------------------------------------------------------------------
%  SEZIONE: diario.tex (Diario della riunione)
% -------------------------------------------------------------------
\newpage
\section{Diario della Riunione}

\subsection{Versionamento}
La discussione si apre analizzando le problematiche relative alla gestione del versionamento dei documenti.
Leonardo Bilato e Giovanni Ponso hanno presentato tre possibili approcci:

\begin{enumerate}
    \item \textbf{Overleaf Condiviso:} Utilizzare un unico documento Overleaf condiviso, accettando un tracciamento delle versioni meno granulare.
    
    \item \textbf{Branch "In Lavorazione" (WIP):} Creare un branch apposito per i documenti in lavorazione. 
    In questa modalità, la versione del documento (es. 1.0.0) si aggiorna solo quando si inizia a lavorare su quel documento. 
    Ogni push successivo \textbf{non} aggiorna la versione. La versione rappresenta un documento "finito" (sebbene temporaneamente). 
    I documenti in lavorazione non verrebbero pubblicati sul sito web.
    
    \item \textbf{Branch Separato con Versioning Semantico Dettagliato:} 
    Creare un branch separato dal \texttt{main} per le modifiche.
        \begin{itemize}
            \item Alla prima modifica della versione 1.0.0, si genera la 1.1.0.
            \item Finché si modifica la stessa \textit{subsection} (oggetto della 1.1.0), si incrementa l'ultima cifra (1.1.1, 1.1.2, etc.).
            \item Quando si passa a una nuova \textit{subsection} (e la precedente è verificata), si passa alla 1.2.0.
            \item Finite tutte le \textit{subsection}, si arriva alla 2.0.0, che viene "pushata" sul \texttt{main} e usata per generare il PDF.
            \item La tabella delle versioni verrebbe aggiornata solo con le versioni del tipo \texttt{x.y.0} (major/minor), ignorando le \texttt{x.y.z} (patch).
            \item Verrebbe implementata una GitHub Action per automatizzare l'aggiornamento:
                \begin{itemize}
                    \item Il file viene rinominato dallo script.
                    \item La tabella delle versioni viene aggiornata automaticamente usando:
                        \item \textbf{Versione:} Quella indicata nel file.
                        \item \textbf{Redattore:} Il committer (o i committer).
                        \item \textbf{Verificatore:} Identificato se scrive "verifica" nel messaggio di commit.
                        \item \textbf{Modifiche:} Il messaggio di commit.
                \end{itemize}
        \end{itemize}
\end{enumerate}

\subsubsection*{Feedback di Giovanni Ponso}
Giovanni Ponso ha espresso perplessità sull'attuale gestione delle versioni:
\begin{itemize}
    \item Non trova molto senso nel modo attuale di gestire le versioni.
    \item Propone di partire dalla 1.0 (invece che 0.1), ma il gruppo nota che creare più righe per la stessa versione (1.0) nella tabella non è fattibile/sensato.
\end{itemize}

\subsubsection*{Conclusione}
Non riuscendo a definire una scelta corretta e univoca, il gruppo delibera di \textbf{chiedere al docente} quale sia la \textit{best practice} consigliata.
Se ne occuperanno Leonardo Bilato e Giovanni Ponso.

\subsection{Prossimi passi verso la RTB}
Si discute la pianificazione delle attività per il prossimo futuro, in vista della RTB.

\begin{itemize}
    \item \textbf{Incontro con l'Azienda:} È stata inviata un'email per richiedere un incontro. È necessario accordarsi al più presto per discutere e definire le modalità di comunicazione:
        \begin{itemize}
            \item \textbf{Asincrona:} (e.g., email, Slack, etc.)
            \item \textbf{Sincrona:} (e.g., Google Meet, Teams, etc.)
        \end{itemize}
    
    \item \textbf{Avvio Attività:} Bisogna iniziare la stesura dei seguenti documenti:
        \begin{itemize}
            \item Analisi dei Requisiti
            \item Piano di Progetto
        \end{itemize}
        
    \item \textbf{Documenti in corso:} Proseguire con l'aggiornamento e la stesura dei documenti:
        \begin{itemize}
            \item Norme di Progetto
            \item Glossario
        \end{itemize}
\end{itemize}


% -------------------------------------------------------------------
%  SEZIONE: decisioni.tex (Decisioni prese)
% -------------------------------------------------------------------
\newpage
\section{Decisioni Prese}

\begin{enumerate}
    \item Non è stata presa una decisione definitiva sulla strategia di versionamento.
    \item Si delibera di consultare il docente per avere indicazioni sulla \textit{best practice} da adottare per il versionamento.
    \item Si procederà a fissare un incontro con l'azienda per definire le modalità di comunicazione e avviare l'analisi dei requisiti.
    \item Iniziare la stesura dei documenti "Analisi dei Requisiti" e "Piano di Progetto".
    \item Continuare l'aggiornamento dei documenti "Norme di Progetto" e "Glossario".
\end{enumerate}

% -------------------------------------------------------------------
%  SEZIONE: todo.tex (Attività da svolgere)
% -------------------------------------------------------------------
\newpage
\section{Attività da Svolgere (To-Do)}

\begin{table}[h!]
\centering
\begin{tabular}{@{}lll@{}}
\toprule
\textbf{Attività} & \textbf{Assegnatario/i} & \textbf{Scadenza} \\
\midrule
Chiedere al docente best practice versionamento & Leonardo Bilato, Giovanni Ponso & Appena possibile \\
Fissare incontro con l'azienda & Tutto il gruppo & Appena possibile \\
Iniziare Analisi dei Requisiti & Leonardo Bilato, Mihaela Romascu & 2025-11-16 \\
Iniziare Piano di Progetto & G. Ponso, F. Zanella, S. Perozzo & 2025-11-16 \\
Continuare Norme di Progetto & Davide Biasuzzi & 2025-11-16 \\
Continuare Glossario & Davide Biasuzzi & 2025-11-16 \\
Stendere verbale riunione & Davide Biasuzzi & 2025-11-11 \\
\bottomrule
\end{tabular}
\caption{Riepilogo task assegnati.}
\end{table}
\end{document}