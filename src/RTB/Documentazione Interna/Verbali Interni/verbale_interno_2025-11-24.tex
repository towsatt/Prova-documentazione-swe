\documentclass[a4paper, 11pt, oneside]{scrartcl} % Classe KOMA-Script

% --- Pacchetti Fondamentali ---
\usepackage[utf8]{inputenc}      % Codifica UTF-8
\usepackage[T1]{fontenc}         % Font encoding moderno
\usepackage[italian]{babel}      % Lingua italiana 
\usepackage{lmodern}             % Font "Latin Modern"

% --- Grafica e Layout ---
\usepackage{graphicx}            % Per includere immagini
\usepackage{currfile}
\graphicspath{{assets/},{../../../assets/}}

\usepackage[a4paper, top=2.5cm, bottom=3cm, left=2.5cm, right=2.5cm]{geometry} % Margini
\usepackage{fancyhdr}            % Per header e footer personalizzati
\usepackage{microtype}           % Migliora la tipografia
\usepackage[svgnames]{xcolor}    % Colori

% --- Utility ---
\usepackage{booktabs}            % Tabelle più professionali
\usepackage{enumitem}            % Per personalizzare liste
\usepackage{hyperref}            % Rende i link cliccabili
\hypersetup{
    colorlinks=true,
    linkcolor=DarkBlue,
    filecolor=DarkBlue,     
    urlcolor=DarkBlue,
    citecolor=DarkBlue,
    pdftitle={Verbale Interno 2025-11-24 - NightPRO},
    pdfauthor={Gruppo NightPRO},
}

% ===================================================================
%  IMPOSTAZIONE HEADER E FOOTER
% ===================================================================
\pagestyle{fancy}
\fancyhf{} % Pulisce tutti i campi
\fancyhead[L]{NightPRO - Progetto Ingegneria del Software}
\fancyhead[R]{Anno Accademico 2025/2026}
\fancyfoot[C]{\thepage} % Numero di pagina al centro in basso
\renewcommand{\headrulewidth}{0.4pt} % Linea sottile sotto l'header
\renewcommand{\footrulewidth}{0pt}

% ===================================================================
%  INIZIO DEL DOCUMENTO
% ===================================================================
\begin{document}

% -------------------------------------------------------------------
%  SEZIONE: intestazione_titolo.tex
% -------------------------------------------------------------------
\thispagestyle{empty}
\begin{titlepage}
    \centering
    
\begin{figure}
    \centering
    \includegraphics[width=0.4\textwidth]{logo.png}
\end{figure}

    \vfill
    
    {\small UNIVERSITÀ DEGLI STUDI DI PADOVA \par}
    {\small CORSO DI LAUREA IN INFORMATICA (L-31) \par}
    \vspace{0.5cm}
    {\large Corso di Ingegneria del Software \par}
    {\small Anno Accademico 2025/2026 \par}
    
    \vfill
    
    {\Huge \bfseries Verbale di Riunione \par}
    
    \vspace{1cm}
    
    {\Large \itshape Verbale Interno del 24 Novembre 2025 \par} 
    
    \vfill
    
    {\Large \bfseries Gruppo: NightPRO \par}
    \vspace{0.5cm}
    {\large \href{mailto:swe.nightpro@gmail.com}{swe.nightpro@gmail.com} \par}
    
    \vfill
 
    {\large Data: 2025-11-24 \par}

\end{titlepage}

% -------------------------------------------------------------------
%  SEZIONE: indice.tex
% -------------------------------------------------------------------
\newpage
\tableofcontents % Genera l'indice
\pagestyle{fancy} % Riattiva lo stile di pagina da qui in poi

% -------------------------------------------------------------------
%  SEZIONE: informazioni.tex
% -------------------------------------------------------------------
\newpage
\section{Informazioni Generali}

\subsection{Componenti del Gruppo}
Elenco dei membri del gruppo di lavoro NightPRO.
\begin{table}[h!]
\centering
\begin{tabular}{@{}llc@{}}
\toprule
\textbf{Cognome} & \textbf{Nome} & \textbf{Matricola} \\
\midrule
Biasuzzi & Davide & 2111000 \\
Bilato & Leonardo & 2071084 \\
Zanella & Francesco & 2116442 \\
Romascu & Mihaela-Mariana & 2079726 \\
Ogniben & Michele & 2042325 \\
Perozzo & Samuele & 2110989 \\
Ponso & Giovanni & 2000558 \\
\bottomrule
\end{tabular}
\caption{Componenti del Gruppo NightPRO.}
\end{table}

\subsection{Dettagli Riunione}
\begin{itemize}
    \item \textbf{Data:} 2025-11-24
    \item \textbf{Ora:} 16:30 - 18:00
    \item \textbf{Luogo:} Google Meet
    \item \textbf{Partecipanti:} Tutti i membri del gruppo
    \item \textbf{Redatto da:} Giovanni Ponso
    \item \textbf{Verificato da:} Leonardo Bilato
    \item \textbf{Versione:} 1.0
\end{itemize}


% -------------------------------------------------------------------
%  SEZIONE: odg.tex (Ordine del Giorno)
% -------------------------------------------------------------------
\newpage
\section{Ordine del Giorno (Agenda)}
\begin{itemize}
    \item[1.] Comunicazioni al Responsabile: Richieste pendenti a Ergon Informatica
    \item[2.] Analisi della prima bozza di "Analisi dei Requisiti"
    \item[3.] Identificazione dubbi da discutere con il Proponente
    \item[4.] Dimostrazione strumento di lavoro: Overleaf
\end{itemize}

% -------------------------------------------------------------------
%  SEZIONE: diario.tex (Diario della riunione)
% -------------------------------------------------------------------
\newpage
\section{Diario della Riunione}

\subsection{Comunicazioni al Responsabile}
La riunione si è aperta con un aggiornamento diretto al Responsabile di periodo, Giovanni Ponso. Facendo riferimento a quanto concordato nel \textbf{Verbale Esterno del 12 Novembre 2025}, è emersa la necessità di sollecitare l'azienda proponente su due punti fondamentali rimasti in sospeso:
\begin{itemize}
    \item \textbf{Contatto Telegram:} Durante l'incontro del 12/11 era stata definita la possibilità di utilizzare Telegram per le comunicazioni asincrone, ma il contatto specifico non è ancora stato fornito.
    \item \textbf{Schema JSON e Caso Studio:} L'azienda si era impegnata a fornire un caso di studio e lo schema JSON dell'output nei primi giorni della settimana successiva alla riunione esterna. Tali risorse non sono ancora pervenute e sono bloccanti per la corretta stesura dei requisiti.
\end{itemize}

\subsection{Analisi della prima bozza "Analisi dei Requisiti"}
L'oggetto principale della riunione è stata l'analisi di gruppo della prima bozza del documento "Analisi dei Requisiti".
La discussione è avvenuta punto per punto, verificando la coerenza con quanto richiesto dal capitolato SmartOrder. Questa analisi ha permesso di evidenziare alcune lacune interpretative.

\subsection{Dubbi e pianificazione incontro con il Proponente}
Durante l'analisi sono emersi tre dubbi principali relativi alle funzionalità e all'architettura del sistema, da chiarire con il referente:
\begin{enumerate}
    \item \textbf{Conferma Ordini:} È necessario capire se gli ordini generati debbano essere esplicitamente confermati dall'utente prima dell'invio definitivo.
    \item \textbf{Perimetro sviluppo WebApp:} Si intende verificare che il requisito obbligatorio riguardi esclusivamente l'interfaccia per l'operatore, rendendo di fatto opzionale il lato client. Di conseguenza, è necessario definire puntualmente quali canali (es. email, messaggistica) l'utente finale utilizzerà per l'inoltro degli ordini.
    \item \textbf{Regole Aziendali:} È necessario un approfondimento sul significato di "regole aziendali e controlli preliminari" citati al punto 5 dell'architettura proposta nel capitolato.
\end{enumerate}
\textbf{Decisione:} Il gruppo ha deciso di richiedere un incontro (call su Zoom) al referente Gianluca Carlesso, proponendo come disponibilità Venerdì mattina. In caso di indisponibilità si propone un confronto asincrono.

\subsection{Adozione di Overleaf}
In chiusura, si è tenuta una breve dimostrazione pratica sull'utilizzo di Overleaf per la stesura della documentazione, evidenziando l'ottima gestione della scrittura concorrente e la facilità di centralizzazione dei documenti, con l'obiettivo di superare le difficoltà iniziali dovute alla scarsa familiarità con lo strumento.

\subsection{Varie ed Eventuali}
\textbf{Piano di Progetto:} Si segnala lo slittamento della scadenza per la stesura dei periodi (1, 2, 3 e 4) del Piano di Progetto, originariamente assegnata nella riunione precedente. Il termine è stato riprogrammato per \textbf{Venerdí 28 Novembre 2025} per permettere un allineamento più preciso tra le varie sezioni.

% -------------------------------------------------------------------
%  SEZIONE: decisioni.tex (Decisioni prese)
% -------------------------------------------------------------------
\newpage
\section{Decisioni Prese}

\begin{enumerate}
    \item Si procederà a contattare Ergon Informatica per richiedere il contatto Telegram e il JSON del caso studio (come da accordi del 12/11).
    \item Verrà inviata una richiesta di appuntamento al referente Gianluca Carlesso per Venerdì mattina (via Zoom) per chiarire i nuovi dubbi emersi.
    \item Overleaf è stato ufficialmente adottato come strumento unico per la redazione collaborativa dei documenti.
    \item Si prosegue attivamente con la stesura e la discussione interna dell'Analisi dei Requisiti, documento attualmente in piena fase di elaborazione e definizione dei contenuti.
\end{enumerate}

% -------------------------------------------------------------------
%  SEZIONE: todo.tex (Attività da svolgere)
% -------------------------------------------------------------------
\newpage
\section{Attività da Svolgere (To-Do)}

\begin{table}[h!]
\centering
\begin{tabular}{@{}lll@{}}
\toprule
\textbf{Attività} & \textbf{Assegnatario/i} & \textbf{Scadenza} \\
\midrule
Richiesta contatto Telegram e JSON & Giovanni Ponso & 2025-11-25 \\
Richiesta incontro Zoom & Giovanni Ponso & 2025-11-25 \\
Stesura Verbale Riunione & Giovanni Ponso & 2025-11-25 \\
Verifica Verbale Riunione & Leonardo Bilato & 2025-11-26 \\
Continuazione Analisi dei Requisiti & Davide Biasuzzi, Samuele Perozzo & 2025-11-28 \\
Stesura Periodi PdP (1, 2, 3, 4) & Romascu, Ogniben, Perozzo, Zanella & 2025-11-28 \\
Configurazione e migrazione su Overleaf & Tutto il gruppo & Immediata \\
\bottomrule
\end{tabular}
\caption{Riepilogo task assegnati.}
\end{table}
\end{document}