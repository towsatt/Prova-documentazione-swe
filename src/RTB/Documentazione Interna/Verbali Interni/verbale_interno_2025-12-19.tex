\documentclass[a4paper, 11pt, oneside]{scrartcl}
\usepackage[utf8]{inputenc}
\usepackage[T1]{fontenc}
\usepackage[italian]{babel}
\usepackage{lmodern}
\usepackage{graphicx}
\usepackage{currfile}
\usepackage{array}
\graphicspath{{assets/}{../assets/}{../../assets/}{../../../assets/}}
\usepackage[a4paper, top=2.5cm, bottom=3cm, left=2.5cm, right=2.5cm]{geometry}
\usepackage{fancyhdr}
\usepackage{microtype}
\usepackage[svgnames]{xcolor}
\usepackage{booktabs}
\usepackage{enumitem}
\usepackage{hyperref}
\hypersetup{
    colorlinks=true,
    linkcolor=DarkBlue,
    filecolor=DarkBlue,
    urlcolor=DarkBlue,
    citecolor=DarkBlue,
    pdftitle={Verbale Interno 2025-12-23 - NightPRO},
    pdfauthor={Gruppo NightPRO},
}
\pagestyle{fancy}
\fancyhf{}
\fancyhead[L]{NightPRO - Progetto Ingegneria del Software}
\fancyhead[R]{Anno Accademico 2025/2026}
\fancyfoot[C]{\thepage}
\renewcommand{\headrulewidth}{0.4pt}
\renewcommand{\footrulewidth}{0pt}

% ------ FRONTESPIZIO -----------------------------------------
\begin{document}
\thispagestyle{empty}
\begin{titlepage}
    \centering
\begin{figure}
    \centering
    % Assicurati che il logo sia presente nella cartella assets
    \includegraphics[width=0.4\textwidth]{logo.png}
\end{figure}
    \vfill
    {\small UNIVERSITÀ DEGLI STUDI DI PADOVA \par}
    {\small CORSO DI LAUREA IN INFORMATICA (L-31) \par}
    \vspace{0.5cm}
    {\large Corso di Ingegneria del Software \par}
    {\small Anno Accademico 2025/2026 \par}
    \vfill
    {\Huge \bfseries Verbale di Riunione \par}
    \vspace{1cm}

    {\Large \itshape Verbale Interno del 19 Dicembre 2025 \par}
    \vfill
    {\Large \bfseries Gruppo: NightPRO \par}
    \vspace{0.5cm}
    {\large \href{mailto:swe.nightpro@gmail.com}{swe.nightpro@gmail.com} \par}
    \vfill

    {\large Data: 2025-12-19 \par}

\end{titlepage}

% ------ INDICE -----------------------------------------
\newpage
\tableofcontents
\pagestyle{fancy}

% ------ INFORMAZIONI GENERALI -----------------------------------------
\newpage
\section{Informazioni Generali}
\subsection{Componenti del Gruppo}
Elenco dei membri del gruppo di lavoro NightPRO.
\begin{table}[h!]
\centering
\begin{tabular}{@{}llc@{}}
\toprule
\textbf{Cognome} & \textbf{Nome} & \textbf{Matricola} \\
\midrule
Biasuzzi & Davide & 2111000 \\
Bilato & Leonardo & 2071084 \\
Zanella & Francesco & 2116442 \\
Romascu & Mihaela-Mariana & 2079726 \\
Ogniben & Michele & 2042325 \\
Perozzo & Samuele & 2110989 \\
Ponso & Giovanni & 2000558 \\
\bottomrule
\end{tabular}
\caption{Componenti del Gruppo NightPRO.}
\end{table}

% ------ DETTAGLI RIUNIONE -----------------------------------------
\subsection{Dettagli Riunione}
\begin{itemize}
    \item \textbf{Data:} 2025-12-19
    \item \textbf{Ora:} 10:00 - 11:30
    \item \textbf{Luogo:} Google Meet
    \item \textbf{Partecipanti:} Samuele Perozzo, Davide Biasuzzi, Mihaela-Mariana Romascu, Giovanni Ponso, Michele Ogniben, Francesco Zanella, Leonardo Bilato.
    \item \textbf{Redatto da:} Mihaela-Mariana Romascu
    \item \textbf{Verificato da:} Leonardo Bilato
    \item \textbf{Versione:} 1.0
\end{itemize}


% ------ ORDINI DEL GIORNO -----------------------------------------
\newpage
\section{Ordine del Giorno (Agenda)}
\begin{itemize}
    \item[1.] Presentazione della bozza del Proof of Concept (PoC).
    \item[2.] Criticità e componenti mancanti (database, template JSON, chiavi LLM) e sollecitazione al proponente.
    \item[3.] Aspettative, timing e pianificazione del proseguimento.
    \item[4.] Riunione di formazione tecnica.
    \item[5.] Miglioramenti tecnici sul PoC e pianificazione dello Sprint 7.
\end{itemize}

% ------ DIARIO DELLA RIUNIONE -----------------------------------------
\newpage
\section{Diario della Riunione}

\subsection{Presentazione Proof of Concept}
Michele ha presentato al gruppo una bozza del futuro sito e una prototipo della funzionalità di chat, realizzati utilizzando le tecnologie React e JavaScript. La presentazione ha illustrato l'idea fondamentale del sistema, fornendo ai presenti una visione concreta dell'interfaccia utente e dell'architettura di base prevista per il progetto. Il prototipo ha ricevuto riscontri positivi dal team, confermando la fattibilità tecnica della soluzione proposta.

\subsection{Criticità: Mancanza Componenti dal Proponente}
Durante la discussione, il gruppo ha identificato un problema critico che ostacola l'avanzamento dei lavori: l'azienda proponente non ha fornito componenti essenziali per lo sviluppo del Proof of Concept. In particolare, sono stati richiesti:
\begin{itemize}
    \item Le specifiche e l'accesso al database;
    \item Il template JSON;
    \item Le chiavi e le credenziali necessarie per il fine-tuning del modello LLM.
\end{itemize}

La mancanza di una risposta da parte del proponente rappresenta un ostacolo significativo. È stato quindi assegnato a \textbf{Davide Biasuzzi} il compito di sollecitare ulteriormente il proponente attraverso i canali di messaggistica istantanea (Telegram e WhatsApp), enfatizzando l'urgenza della situazione.

Nel caso in cui il proponente non fornisca i componenti richiesti entro i tempi previsti, il gruppo sarà costretto a discutere lo spostamento della milestone RTB attualmente prevista per il 16 Gennaio 2026.

\subsection{Aspettative e Timing}
In attesa del reperimento delle risorse richieste, il gruppo ha stabilito che:
\begin{itemize}
    \item L'implementazione del Proof of Concept dovrà proseguire parallelamente, affrontando gli aspetti completabili anche in assenza dei componenti critici;
    \item La data prevista per il completamento della fase di RTB rimane il 16 Gennaio 2026, a condizione che le risorse vengano fornite entro i tempi previsti;
    \item Nel caso in cui il proponente non fornisca i componenti richiesti, sarà necessario inviare un messaggio di chiarimento sulla revisione della scadenza.
\end{itemize}

\subsection{Riunione di Formazione Tecnica}
È stata già programmata una riunione di formazione per il 12 Gennaio 2026. In merito a questa, il gruppo deciderà nei prossimi incontri sulla base dello stato di avanzamento dei lavori e della disponibilità del team.


% ------ DECISIONI PRESE -----------------------------------------
\section{Decisioni Prese}
\begin{enumerate}
    \item Approvazione della presentazione del Proof of Concept realizzato da Michele Ogniben.
    \item Incarico a Davide Biasuzzi di sollecitare il proponente per il reperimento delle componenti mancanti.
\end{enumerate}


\newpage
% ------ PIANIFICAZIONE DI PERIODO -----------------------------------------
\section{Attività da Svolgere (Sprint 7)}
Le seguenti attività sono pianificate per lo Sprint 7.
\begin{table}[h!]
\centering
\begin{tabular}{@{}p{10cm}l@{}}
\toprule
\textbf{Attività} & \textbf{Assegnatario/i} \\
\midrule
Pianificazione dell'architettura del database & Progettista \\
Implementazione del database & Programmatore \\
Implementazione nel PoC del modello LLM & Programmatore \\
Fine-tuning del modello & Programmatore \\
Valutazione della migrazione a Next.js & Programmatore \\
Aggiunta dei diagrammi nell'Analisi dei Requisiti & Analista \\
Attività di coordinamento e stesura verbali & Responsabile \\
Verifica delle prestazioni del sistema & Verificatore \\
\bottomrule
\end{tabular}
\caption{Task assegnati per lo Sprint 7.}
\end{table}

% tabella di rotazione dei ruoli
\section{Definizione dei ruoli (Preventivo Sprint 7)}
Sono stati definiti i ruoli e le relative assegnazioni orarie per il prossimo periodo operativo (Sprint 7).
La ripartizione è stata stabilita come segue:

\begin{table}[h!]
\centering
\begin{tabular}{@{}l l@{}}
\toprule
\textbf{Ruolo} & \textbf{Membro (Ore)} \\
\midrule
Responsabile   & Mihaela-Mariana Romascu (5h) \\
Amministratore & - \\
Analista       & Davide Biasuzzi (5h) \\
Progettista    & Giovanni Ponso (2h) \\
Programmatore  & Giovanni Ponso (2h), Michele Ogniben (5h), Samuele Perozzo (5h),\\
 & Francesco Zanella (5h) \\
Verificatore   & Leonardo Bilato (4h) \\
\bottomrule
\end{tabular}
\caption{Distribuzione ruoli e ore per lo Sprint 7.}
\end{table}
\newpage
% ------ PIANIFICAZIONE DI PERIODO -----------------------------------------

\end{document}
