\documentclass[a4paper, 11pt, oneside]{scrartcl}
\usepackage[utf8]{inputenc}
\usepackage[T1]{fontenc}
\usepackage[italian]{babel}
\usepackage{lmodern}
\usepackage{graphicx}
\usepackage{currfile}
\usepackage{array}
\graphicspath{{assets/}{../assets/}{../../assets/}{../../../assets/}}
\usepackage[a4paper, top=2.5cm, bottom=3cm, left=2.5cm, right=2.5cm]{geometry}
\usepackage{fancyhdr}
\usepackage{microtype}
\usepackage[svgnames]{xcolor}
\usepackage{booktabs}
\usepackage{enumitem}
\usepackage{hyperref}
\hypersetup{
    colorlinks=true,
    linkcolor=DarkBlue,
    filecolor=DarkBlue,
    urlcolor=DarkBlue,
    citecolor=DarkBlue,
    pdftitle={Verbale Interno 2026-01-13 - NightPRO},
    pdfauthor={Gruppo NightPRO},
}
\pagestyle{fancy}
\fancyhf{}
\fancyhead[L]{NightPRO - Progetto Ingegneria del Software}
\fancyhead[R]{Anno Accademico 2025/2026}
\fancyfoot[C]{\thepage}
\renewcommand{\headrulewidth}{0.4pt}
\renewcommand{\footrulewidth}{0pt}

% ------ FRONTESPIZIO -----------------------------------------
\begin{document}
\thispagestyle{empty}
\begin{titlepage}
    \centering
\begin{figure}
    \centering
    % Assicurati che il logo sia presente nella cartella assets
    \includegraphics[width=0.4\textwidth]{logo.png}
\end{figure}
    \vfill
    {\small UNIVERSITÀ DEGLI STUDI DI PADOVA \par}
    {\small CORSO DI LAUREA IN INFORMATICA (L-31) \par}
    \vspace{0.5cm}
    {\large Corso di Ingegneria del Software \par}
    {\small Anno Accademico 2025/2026 \par}
    \vfill
    {\Huge \bfseries Verbale di Riunione \par}
    \vspace{1cm}

    {\Large \itshape Verbale Interno del 13 Gennaio 2026 \par}
    \vfill
    {\Large \bfseries Gruppo: NightPRO \par}
    \vspace{0.5cm}
    {\large \href{mailto:swe.nightpro@gmail.com}{swe.nightpro@gmail.com} \par}
    \vfill

    {\large Data: 2026-01-13 \par}

\end{titlepage}

% ------ INDICE -----------------------------------------
\newpage
\tableofcontents
\pagestyle{fancy}

% ------ INFORMAZIONI GENERALI -----------------------------------------
\newpage
\section{Informazioni Generali}
\subsection{Componenti del Gruppo}
Elenco dei membri del gruppo di lavoro NightPRO.
\begin{table}[h!]
\centering
\begin{tabular}{@{}llc@{}}
\toprule
\textbf{Cognome} & \textbf{Nome} & \textbf{Matricola} \\
\midrule
Biasuzzi & Davide & 2111000 \\
Bilato & Leonardo & 2071084 \\
Zanella & Francesco & 2116442 \\
Romascu & Mihaela-Mariana & 2079726 \\
Ogniben & Michele & 2042325 \\
Perozzo & Samuele & 2110989 \\
Ponso & Giovanni & 2000558 \\
\bottomrule
\end{tabular}
\caption{Componenti del Gruppo NightPRO.}
\end{table}

% ------ DETTAGLI RIUNIONE -----------------------------------------
\subsection{Dettagli Riunione}
\begin{itemize}
    \item \textbf{Data:} 2026-01-13
    \item \textbf{Ora:} 18:30 - 21:00
    \item \textbf{Luogo:} Online Meet
    \item \textbf{Partecipanti:} Francesco Zanella, Samuele Perozzo, Michele Ogniben, Giovanni Ponso, Davide Biasuzzi, Leonardo Bilato, Mihaela-Mariana Romascu.
    \item \textbf{Assenti:} Nessuno
    \item \textbf{Redatto da:} Davide Biasuzzi
    \item \textbf{Verificato da:} Giovanni Ponso
    \item \textbf{Versione:} 1.0
\end{itemize}


% ------ ORDINI DEL GIORNO -----------------------------------------
\newpage
\section{Ordine del Giorno (Agenda)}
\begin{itemize}
    \item[1.] Stato di avanzamento collegamento backend e frontend.
    \item[2.] Gestione infrastruttura e chiavi API OpenAI.
    \item[3.] Scelta tecnica tra Fine-tuning e RAG (Retrieval-Augmented Generation).
    \item[4.] Analisi dei requisiti e design dell'interfaccia (Analisi finale).
    \item[5.] Preparazione domande per l'incontro con Ergon/Carlesso.
\end{itemize}

% ------ DIARIO DELLA RIUNIONE -----------------------------------------
\newpage
\section{Diario della Riunione}

\subsection{Stato Tecnico e Architettura del Sistema}
Il gruppo ha verificato lo stato di avanzamento del collegamento tra backend (FastAPI) e frontend (React). Michele Ogniben si è occupato della parte iniziale di backend con integrazione LLM, mentre Samuele Perozzo e Francesco Zanella si sono occupati della connessione backend-frontend-database.

In particolare, è stata discussa l'implementazione della logica di recupero dati fondamentale per il sistema RAG (Retrieval-Augmented Generation). Durante questa fase è emersa la necessità di implementare un meccanismo di ricerca "fuzzy" per gestire eventuali errori di battitura da parte degli utenti. Il gruppo ha approfondito l'utilizzo dell'algoritmo \textbf{Soundex}, che permette di identificare parole foneticamente simili, migliorando così l'esperienza utente e la robustezza del sistema.

L'infrastruttura di backend è stata realizzata utilizzando \textbf{Docker} e \textbf{FastAPI}, garantendo modularità e facilità di deployment. Tuttavia, durante la fase di test è emersa una criticità relativa alla chiave API OpenAI fornita dall'azienda proponente: il credito associato alla chiave risulta esaurito, impedendo il completamento dei test. Temporaneamente, un membro del gruppo ha messo a disposizione la propria chiave personale con credito residuo per consentire lo sviluppo e i test necessari. Questa problematica sarà discussa nella prossima riunione esterna con l'azienda per richiedere una nuova chiave funzionante.

\subsection{Scelta Architetturale: RAG vs Fine-tuning}
Una parte significativa della riunione è stata dedicata alla discussione della scelta tecnica tra \textbf{Fine-tuning} e \textbf{RAG} (Retrieval-Augmented Generation) per l'implementazione del sistema di intelligenza artificiale.

Il gruppo ha analizzato le specifiche del capitolato e ha rilevato che non vi è alcuna menzione esplicita riguardo all'obbligo di utilizzare il fine-tuning del modello linguistico. Dopo un'attenta valutazione, il team ha concluso che il fine-tuning non sia la soluzione più adatta al caso d'uso specifico del progetto, per le seguenti motivazioni:
\begin{itemize}
    \item \textbf{Dati insufficienti:} il fine-tuning richiede un dataset molto ampio e ben strutturato, di cui il gruppo non dispone attualmente;
    \item \textbf{Costi elevati:} il processo di fine-tuning comporta costi computazionali e temporali significativi, non giustificati dalle necessità del progetto;
    \item \textbf{Inadeguatezza al dominio:} per cataloghi aziendali specifici e in continua evoluzione, il RAG risulta più flessibile e manutenibile, permettendo di aggiornare le informazioni senza dover riaddestrare il modello.
\end{itemize}

È stata quindi presa la \textbf{decisione unanime} di utilizzare esclusivamente l'approccio RAG, che permette al sistema di recuperare informazioni contestuali dal database aziendale e fornirle al modello linguistico per generare risposte accurate e aggiornate.

Durante la discussione sono stati approfonditi anche i concetti di \textbf{token} ed \textbf{embedding vettoriale}, fondamentali per ottimizzare le performance di sistema.

È stato deciso di inserire questa scelta tecnica tra le domande da porre all'azienda proponente Ergon durante la prossima riunione esterna, per ottenere conferma e validazione dell'approccio scelto.

\subsection{Infrastruttura e Hosting: Gestione Railway e Repository}
Il gruppo ha discusso sul confermare \textbf{Railway} come piattaforma di hosting della webapp, e che avremmo chiesto conferma all'azienda proponente durante la prossima riunione esterna.

Un'altra criticità emersa riguarda la pulizia del database: sono stati identificati ordini e articoli con chiavi esterne mancanti o inconsistenti, che devono essere rimossi per garantire l'integrità referenziale e il corretto funzionamento del sistema. Giovanni Ponso e Samuele Perozzo sono stati incaricati di procedere con la pulizia del database e la configurazione corretta di Railway.

Inoltre, è stata sottolineata l'importanza di implementare un sistema di \textbf{logging delle chiamate API} per facilitare il debugging, specialmente per gestire eventuali errori 429 (rate limit) che potrebbero verificarsi durante l'utilizzo intensivo delle API di OpenAI.

\subsection{Analisi dei Requisiti e Design dell'Interfaccia Utente}
Mihaela-Mariana Romascu e Leonardo Bilato hanno presentato i mockup dell'interfaccia utente realizzati con \textbf{Penpot}, illustrando sia l'interfaccia destinata al cliente finale sia quella per l'operatore aziendale.

Durante la presentazione sono emerse diverse proposte di miglioramento e funzionalità da integrare:

\paragraph{Gestione degli ordini ambigui}
È stato discusso come gestire l'ambiguità negli ordini ricevuti dal sistema. Il gruppo ha deciso di implementare una \textbf{percentuale di confidenza} (affidabilità) per ogni ordine, visualizzata nell'interfaccia dell'operatore. Questo permetterà di identificare rapidamente gli ordini che necessitano di revisione manuale.

\paragraph{Interfaccia operatore}
Per migliorare l'usabilità dell'interfaccia dell'operatore, sono state decise le seguenti modifiche:
\begin{itemize}
    \item \textbf{Colonna ID univoca} per ogni ordine nella tabella principale, facilitando l'identificazione e la tracciabilità degli ordini;
    \item \textbf{Sistema di filtri per data} con ordinamento crescente/decrescente, implementato tramite icone dedicate invece di una barra di ricerca complessa;
    \item \textbf{Barra di ricerca dinamica} per cercare ordini specifici per cliente o prodotto;
    \item \textbf{Pulsante di richiesta assistenza umana} per permettere al cliente di segnalare esplicitamente situazioni in cui il sistema automatico non è sufficiente.
\end{itemize}

Mihaela-Mariana Romascu è stata incaricata di aggiornare i mockup includendo le modifiche discusse, in particolare l'aggiunta della colonna ID e dei filtri per data.

\subsection{Preparazione della Riunione Esterna con Ergon}
Il gruppo ha raccolto e discusso i dubbi e le domande da porre durante la riunione esterna prevista con i rappresentanti dell'azienda Ergon e il professor Carlesso.

Le principali questioni da chiarire riguardano:
\begin{itemize}
    \item \textbf{Dati mancanti nel dataset:} verificare se sia possibile ottenere informazioni aggiuntive quali prezzi, quantità disponibili e disponibilità dei prodotti, attualmente non presenti nel dataset fornito;
    \item \textbf{Validazione dell'approccio RAG:} confermare la scelta tecnica di utilizzare RAG invece del fine-tuning, spiegando le motivazioni tecniche e ottenendo l'approvazione dell'azienda;
    \item \textbf{Processo di assistenza dell'operatore:} discutere del processo di assistenza dell'operatore all'utente;
    \item \textbf{Infrastruttura di deploy:} richiedere informazioni sui server aziendali per pianificare il deployment finale del sistema in produzione.
\end{itemize}

Per facilitare la raccolta e la sincronizzazione delle domande tra tutti i membri del team, è stato deciso di creare un \textbf{file Google Doc condiviso} dove ciascun membro potrà inserire le proprie domande prima della riunione. Questo permetterà di avere una visione completa e organizzata di tutti i punti da discutere.

% ------ DECISIONI PRESE -----------------------------------------
\newpage
\section{Decisioni Prese}
\begin{enumerate}
    \item Adozione dell'approccio \textbf{RAG} (Retrieval-Augmented Generation) come soluzione tecnica principale, escludendo il fine-tuning.
    \item Pulizia del database da ordini e articoli con chiavi esterne inconsistenti.
    \item Futura implementazione di un sistema di logging per le chiamate API.
    \item Aggiunta nell'interfaccia operatore di:
    \begin{itemize}
        \item Percentuale di confidenza per identificare ordini ambigui;
        \item Filtri per data con ordinamento cronologico tramite icone dedicate;
        \item Barra di ricerca dinamica per cliente/prodotto;
        \item Pulsante per richiesta assistenza umana.
    \end{itemize}
    \item Creazione di un file Google Doc condiviso per sincronizzare le domande prima della riunione esterna.
    \item Discussione con Ergon nella prossima riunione esterna riguardo a:
    \begin{itemize}
        \item Chiave API OpenAI non funzionante;
        \item Validazione dell'approccio RAG;
        \item Dati mancanti nel dataset;
        \item Conferma piattaforma Railway come hosting della webapp;
        \item Informazioni sui server aziendali per il deploy.
    \end{itemize}
\end{enumerate}

% ------ ATTIVITÀ DA SVOLGERE -----------------------------------------
\section{Attività da Svolgere}
Le attività discusse si intendono implicitamente assegnate ai membri che già se ne occupavano, mantenendo la continuità con il lavoro corrente. La presente riunione ha avuto funzione di disambiguazione e allineamento interno in vista dell'incontro con l'azienda previsto per il\textbf{14 Gennaio 2026}. La pianificazione formale delle attività per il prossimo sprint avverrà durante la riunione di fine sprint del \textbf{16 Gennaio 2026}.
\end{document}
